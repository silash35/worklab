\section{PINNs: Physics-Informed Neural Networks}

Nesta seção analisamos o funcionamento das PINNs e como elas se diferenciam das redes neurais artificiais tradicionais.

\subsection{Redes Neurais Artificiais (ANNs)}

As redes neurais artificiais (ANNs) são uma classe de modelos de aprendizado de máquina que foram desenvolvidos tendo o estudos sobre o cérebro e o sistema nervoso como inspiração. Elas funcionam com base em elementos simples, capazes de receber múltiplas entradas, processa-las e retornar uma saída, chamados neurônios \cite{riad_mania_bouchaou_najjar_2004}. Esses neurônios são então conectados de diversas maneiras diferentes, a depender da arquitetura da rede neural projetada \cite{mu_sun_2022}.

\missingfigure{Representação de um neuronio em uma ANN}

Ao ligar muitos neurônios organizados em diferentes camadas, forma-se uma rede neural capaz de resolver problemas complexos \cite{krenker_bester_kos_2011}.

\missingfigure{Representação de uma rede neural}
